Something about accelerating computation of stress analysis.. Take it from my proposal.

Because of this, there has been decades of work attempting to accelerate solid mechanics simulations for the purposes of stress analysis. Such works take two broad forms, solver-based approached (which attempt to improve the linear algebraic operations at the heart of a physics simulator)or discretization-based approaches (which try to reduce the degrees-of-freedom)in a computational mesh. One of he most successful acceleration and well-loved techniques for speeding up simulations is linear modal analysis, a discretization-based technique which performs simulation in a small reduced space constructed from the linear deformation modes of a deformable object. 

However, linear modal analysis still has issues which prevent it from being a panacea to the performance problems of physics-based simulation. First anything beyond the linear modes can be difficult to compute, relying on non-linear modal derivatives. Second, motions which span highly non-linear regions of the configuration space can require large linear spaces to accurately represent. This can have a dire impact on runtimes since system matrices for small reduced spaces become dense meaning that even a small increase in the number of modes can have a hampering effect on performance (consider that fast integration schemes for reduced models typically have $O(r^2)$ to $O(r^4)$ asymptotic behavior where $r$ is the dimension of the reduced space). 

Ideally, one would generate as small a reduced space as necessary directly from simulation data. A strictly linear basis can make this difficult. In this paper we flaunt conventional wisdom and disregard linearity, in favor of non-liner dimensionality reduction techniques. In particular we draw from the field of machine learning and devise a new autoencoder architecture that can efficiently represent non-linear deformations with high fidelity using fewer degrees of freedom than a corresponding linear reduced representation. Furthermore we develop a minimization based time integrator that takes advantage of our reduced representation to produce dynamic simulations of production quality meshes at interactive rates.

The remainder of this paper proceeds as follows: 

..