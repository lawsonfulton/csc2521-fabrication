\documentclass[acmtog, authorversion]{acmart}

\usepackage{booktabs} % For formal tables

% TOG prefers author-name bib system with square brackets

\citestyle{acmauthoryear}
\setcitestyle{square}


\usepackage[ruled]{algorithm2e} % For algorithms
\renewcommand{\algorithmcfname}{ALGORITHM}
\SetAlFnt{\small}
\SetAlCapFnt{\small}
\SetAlCapNameFnt{\small}
\SetAlCapHSkip{0pt}
\IncMargin{-\parindent}

% Metadata Information
\acmJournal{TOG}
\acmVolume{999}
\acmNumber{999}
\acmArticle{999}
\acmYear{2099}
\acmMonth{99}

% Copyright
%\setcopyright{acmcopyright}
%\setcopyright{acmlicensed}
%\setcopyright{rightsretained}
%\setcopyright{usgov}
%\setcopyright{usgovmixed}
%\setcopyright{cagov}
%\setcopyright{cagovmixed}

% DOI
\acmDOI{0000001.0000001_2}

% Paper history
%\received{February 2007}
%\received{March 2009}
%\received[final version]{June 2009}
%\received[accepted]{July 2009}


% Document starts
\begin{document}
% Title portion
\title{Interactive Stress Analysis for CAD Models
	via Machine Learning} 

\author{Lawson Fulton}
%\orcid{1234-5678-9012-3456}
\affiliation{%
  \institution{University of Toronto}
  \streetaddress{27 King's College Cir.}
  \city{Toronto}
  \state{ON}
  \postcode{M5S 2E4}
  \country{Canada}}
\email{lawson@cs.toronto.edu}


\renewcommand\shortauthors{Fulton}

\begin{abstract}
Something about stress analysis. Something about stress analysis. Something about stress analysis. Something about stress analysis. Something about stress analysis. Something about stress analysis. Something about stress analysis. Something about stress analysis. Something about stress analysis. Something about stress analysis. Something about stress analysis. 
\end{abstract}


%
% The code below should be generated by the tool at
% http://dl.acm.org/ccs.cfm
% Please copy and paste the code instead of the example below. 
%





\maketitle

\section{Introduction}

As a new technology, Wireless Sensor Networks (WSNs) has a wide
range of applications \cite{Schulz-17}, including
environment monitoring, smart buildings, medical care, industrial and
military applications. Among them, a recent trend is to develop
commercial sensor networks that require pervasive sensing of both
environment and human beings, for example, assisted living
\cite{Chen-16} and smart homes
\cite{Harvard-01, Adya-01,CROSSBOW}.
% quote
\begin{quote}
	``For these applications, sensor devices are incorporated into human
	cloths \cite{Natarajan-01, Zhou-06, Bahl-02, Adya-01} for monitoring
	health related information like EKG readings, fall detection, and
	voice recognition''.
\end{quote}
While collecting all these multimedia information
\cite{Akyildiz-02} requires a high network throughput, off-the-shelf
sensor devices only provide very limited bandwidth in a single
channel: 19.2\,Kbps in MICA2 \cite{Bahl-02} and 250\,Kbps in MICAz.

In this article, we propose MMSN, abbreviation for Multifrequency
Media access control for wireless Sensor Networks. The main
contributions of this work can be summarized as follows.
% itemize
\begin{itemize}
	\item To the best of our knowledge, the MMSN protocol is the first
	multifrequency MAC protocol especially designed for WSNs, in which
	each device is equipped with a single radio transceiver and
	the MAC layer packet size is very small.
	\item Instead of using pairwise RTS/CTS frequency negotiation
	\cite{Adya-01, Culler-01, Tzamaloukas-01, Zhou-06},
	we propose lightweight frequency assignments, which are good choices
	for many deployed comparatively static WSNs.
	\item We develop new toggle transmission and snooping techniques to
	enable a single radio transceiver in a sensor device to achieve
	scalable performance, avoiding the nonscalable ``one
	control channel + multiple data channels'' design \cite{Natarajan-01}.
\end{itemize}

% Head 1
\section{Previous Work}

% Head 2
\subsection{Frequency Assignment}17

We propose a suboptimal distribution to be used by each node, which is
easy to compute and does not depend on the number of competing
nodes. A natural candidate is an increasing geometric sequence, in
which
% Numbered Equation
\begin{equation}
\label{eqn:01}
P(t)=\frac{b^{\frac{t+1}{T+1}}-b^{\frac{t}{T+1}}}{b-1},
\end{equation}
where $t=0,{\ldots}\,,T$, and $b$ is a number greater than $1$.

In our algorithm, we use the suboptimal approach for simplicity and


% Head 3
\subsubsection{Exclusive Frequency Assignment}


In exclusive frequency assignment, nodes first exchange their IDs
among two communication hops so that each node knows its two-hop
neighbors' IDs. In the second broadcast, each node beacons all
neighbors' IDs it has collected during the first broadcast period.

% Head 4
\paragraph{Eavesdropping}

Even though the even selection scheme leads to even sharing of
available frequencies among any two-hop neighborhood, it involves a
number of two-hop broadcasts. To reduce the communication cost, we
propose a lightweight eavesdropping scheme.

\section{Proposed Solution}
Even though the even selection scheme leads to even sharing of
available frequencies among any two-hop neighborhood, it involves a
number of two-hop broadcasts. To reduce the communication cost, we
propose a lightweight eavesdropping scheme.

\section{Expected Results}
Winning

% Bibliography
\bibliographystyle{ACM-Reference-Format}
\bibliography{proposal-bibliography}

\end{document}
