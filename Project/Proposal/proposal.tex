\documentclass[acmtog, authorversion]{acmart}

\usepackage{booktabs} % For formal tables

% TOG prefers author-name bib system with square brackets

\citestyle{acmauthoryear}
\setcitestyle{square}


\usepackage[ruled]{algorithm2e} % For algorithms
\renewcommand{\algorithmcfname}{ALGORITHM}
\SetAlFnt{\small}
\SetAlCapFnt{\small}
\SetAlCapNameFnt{\small}
\SetAlCapHSkip{0pt}
\IncMargin{-\parindent}

% Metadata Information
\acmJournal{TOG}
\acmVolume{999}
\acmNumber{999}
\acmArticle{999}
\acmYear{2099}
\acmMonth{99}

% Copyright
%\setcopyright{acmcopyright}
%\setcopyright{acmlicensed}
%\setcopyright{rightsretained}
%\setcopyright{usgov}
%\setcopyright{usgovmixed}
%\setcopyright{cagov}
%\setcopyright{cagovmixed}

% DOI
\acmDOI{0000001.0000001_2}

% Paper history
%\received{February 2007}
%\received{March 2009}
%\received[final version]{June 2009}
%\received[accepted]{July 2009}


% Document starts
\begin{document}
% Title portion
\title{Interactive Stress Analysis for CAD Models
	via Machine Learning} 

\author{Lawson Fulton}
%\orcid{1234-5678-9012-3456}
\affiliation{%
  \institution{University of Toronto}
  \streetaddress{27 King's College Cir.}
  \city{Toronto}
  \state{ON}
  \postcode{M5S 2E4}
  \country{Canada}}
\email{lawson@cs.toronto.edu}


\renewcommand\shortauthors{Fulton}

\begin{abstract}

In this project we propose a method for exploring the design space of parametric CAD models with real-time visualization of stress analysis under small deformations. In particular, we propose precomputation through the training of an artificial neural network (ANN) for one of two ways of accelerating stress field computation. The first is to predict a stress field directly given CAD model parameters and external forces. The second is to use an autoencoder to create a non-linear reduced space in which to compute the stress analysis.

\end{abstract}


%
% The code below should be generated by the tool at
% http://dl.acm.org/ccs.cfm
% Please copy and paste the code instead of the example below. 
%





\maketitle

\section{Introduction}
CAD modelling tools provide an expressive interface for the design of shapes for 3D fabrication. In particular, they allow designers to create parameterized shape families, which facilitate the quick exploration of many shapes intended for a particular use case. Often, the intended use of these shapes requires understandings of their behaviour under load. Structural stress analysis computed on a tetrahedral mesh by FEM is the most common way to achieve this goal. However, to get accurate results, these meshes can require very fine discretizations, which take multiple seconds, to minutes to solve even on today's hardware.


%\begin{itemize}
%	\item To 
%\end{itemize}

% Head 1
\section{Previous Work}

There have been several attempts in recent literature to speed up stress analysis for shape design through pre-computation.
% Head 2
\subsection{Sampling and Interpolation}

The approach used in \cite{Schulz-17} is to do a full stress analysis at multiple points within the parameter space, and use a clever interpolation scheme to produce intermediary results. This approach has the benefit of handling varying mesh topologies, but requires storing $2^k$ meshes where $k$ is the number of model parameters, thus requiring large amounts of storage for complex designs.



% Head 3
\subsection{Reduced Space Computation}

Another approach, as in \cite{Chen-16}, is to use a small linear basis to generate the stress field and do a complete solve within this reduced space. This approach avoids storage of meshes, but has the drawback of only working for fixed mesh topologies. Therefore it is unsuitable for typical CAD models which have varying mesh topologies for each choice of design parameters.

\section{Proposed Solution}

We propose two methods based on techniques from machine learning which mirror the previous approaches, while hopefully avoiding their shortcomings.

\subsection{Direct Stress Field Prediction}

The most straight forward approach is to calculate an approximation of the stress field for a shape directly given the parameters and the external forces. This can be done through the use of an ANN and learning by stochastic gradient descent (SGD). The main challenge using this approach will be choosing the representation of shape geometry on which to do learning. Since each parameter set may correspond to a different mesh topology or size, we must use a representation which is invariant to different mesh parameterizations. One solution is to adapt surface networks \cite{Krostikov-17} to work for tetrahedral meshes. Another is to simply use a fixed size representation by voxelizing the domain and using conventional convolutional neural network techniques \cite{Wang-17}.

\subsection{Autoencoder Dimensionality Reduction}
Another potentially more sophisticated approach is to replicate the results of \cite{Schulz-17} but using an autoencoder framework in place of principal component analysis for stress subspace construction. However, it is uncertain whether their approach will generalize easily to a nonlinear subspace, and futhermore we would still have to address the issue of varying mesh topology. For these reasons, this approach is only of secondary interest, and requires further investigation.

\section{Expected Results}

It is too early to make a confident prediction about the results of this investigation. However, we hope to produce a system of precomputation that will allow the realtime exploration of parameterized CAD models and their associated stress charachteristics. In the ideal case, this system will either consume less memory, perform faster, or give more accurate results than either of the previous approaches.

% Bibliography
\bibliographystyle{ACM-Reference-Format}
\bibliography{proposal-bibliography}

\end{document}
